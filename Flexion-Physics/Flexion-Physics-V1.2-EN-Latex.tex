\documentclass[11pt]{article}

% -----------------------------
% Packages
% -----------------------------
\usepackage[a4paper,margin=1in]{geometry}
\usepackage{amsmath,amssymb,amsthm}
\usepackage{mathtools}
\usepackage{hyperref}
\usepackage{enumitem}
\usepackage{authblk}

% -----------------------------
% Hyperref setup
% -----------------------------
\hypersetup{
  colorlinks=true,
  linkcolor=blue,
  citecolor=blue,
  urlcolor=blue
}

% -----------------------------
% Theorem environments
% -----------------------------
\theoremstyle{definition}
\newtheorem{definition}{Definition}
\newtheorem{axiom}{Axiom}
\newtheorem{theorem}{Theorem}
\newtheorem{remark}{Remark}

% -----------------------------
% Core notation
% -----------------------------
\newcommand{\X}{X}
\newcommand{\Xd}{X_{\Delta}}
\newcommand{\Xp}{X_{\Phi}}
\newcommand{\Xm}{X_{M}}
\newcommand{\Xk}{X_{\kappa}}

\newcommand{\Pphys}{P_{\mathrm{phys}}}

% -----------------------------
% Title / Author
% -----------------------------
\title{\textbf{Flexion Physics V1.2}\\
Structural Origin of Physical Time, Energy, and Singularity Breakdown}

\author[1]{Maryan Bogdanov}
\affil[1]{Independent Researcher}
\date{\today}

\begin{document}
\maketitle

% -----------------------------
% Abstract
% -----------------------------
\begin{abstract}
Flexion Physics V1.2 presents a structural theory in which physical laws,
quantities, and spacetime itself arise as projections of a deeper living
structural state \(X = (\Delta, \Phi, M, \kappa)\).
The theory does not modify or replace established physical models.
Instead, it defines the domain of physical validity, the origin of physical time,
energy, and forces, and the structural reason for the breakdown of physics at
singularities.

Physical observables are shown to exist only while structural viability
\(\kappa > 0\). When \(\kappa = 0\), physics does not diverge but becomes undefined
as a projection domain. This resolves classical and quantum singularities as
projection failures rather than physical infinities.

Flexion Physics establishes a strict separation between structure and physics,
providing a consistent ontological foundation compatible with irreversibility,
collapse theory, and the Flexion Framework.
\end{abstract}

\tableofcontents

% -----------------------------
\section{Introduction}

Contemporary physics provides highly successful mathematical descriptions of
natural phenomena, yet it remains silent on a fundamental question: under what
conditions does physics itself exist? Classical and modern physical theories
typically assume the existence of spacetime, energy, forces, and time as given,
and focus on describing their dynamics within a fixed ontological domain.

Flexion Physics V1.2 addresses a different level of inquiry. It does not propose
new physical laws, modify existing equations, or compete with established
theories such as classical mechanics, relativity, or quantum theory. Instead, it
investigates the structural preconditions under which any physical description
is meaningful at all.

The central premise of Flexion Physics is that physical reality is not
fundamental, but projected. Physical quantities and laws arise as observable
manifestations of a deeper living structural state
\[
X = (\Delta, \Phi, M, \kappa),
\]
defined by the Flexion Framework. Physics is treated as a projection domain,
not as the underlying ontology.

Within this approach, structural variables are not identified with physical
entities: deformation is not matter, structural energy is not physical energy,
structural memory is not entropy, and structural viability is not physical
existence. Physical concepts emerge only through a projection from structure and
inherit strict limitations from it.

A key consequence of this view is that physics exists only while structural
viability remains positive. When structural viability collapses, physics does
not diverge, explode, or become singular; it simply ceases to be defined. This
provides a structural resolution of physical singularities as failures of
projection rather than physical infinities.

The goal of this paper is to formalize this perspective. Flexion Physics V1.2
defines the origin of physical time, energy, and forces as structural
projections, establishes the domain of validity of physical laws, and clarifies
the structural meaning of singularities, irreversibility, and the breakdown of
physical description. The theory is explicitly compatible with existing physics
and makes no empirical or predictive claims beyond its ontological scope.

% -----------------------------
\section{Structural Ontology}

Flexion Physics is grounded in a structural ontology that precedes and constrains
all physical description. This ontology is defined by the Flexion Framework and
does not assume the existence of physical spacetime, matter, energy, or forces as
primitive entities. Instead, it introduces a living structural state whose
evolution gives rise to physical observables only through projection.

\subsection{The Living Structural State}

The fundamental ontological entity is the living structural state
\[
X = (\Delta, \Phi, M, \kappa),
\]
where each component has a strictly structural meaning.
\(\Delta\) represents structural deformation, \(\Phi\) represents structural
energy or tension, \(M\) represents accumulated and irreversible structural
memory, and \(\kappa\) represents structural viability.

The living structural state is not a physical system and does not exist in
spacetime. It is not composed of particles, fields, or forces, and it is not
governed by physical laws. Physical quantities arise only as projections of this
state and have no independent existence outside the projection domain.

\subsection{Irreversibility and Structural Time}

Structural time is not an independent dimension but a consequence of structural
memory. The monotonic accumulation of \(M\) defines an intrinsic ordering of
structural states that cannot be reversed. This ordering constitutes structural
time.

Because structural memory is irreversible, structural time possesses a built-in
arrow. This arrow does not originate from entropy, thermodynamics, or statistical
mechanics, but from the fundamental impossibility of erasing accumulated
structural memory.

Physical time, when it exists, inherits this irreversibility through projection.
The arrow of physical time is therefore a derived property, not a fundamental
assumption.

\subsection{Viability and Structural Termination}

Structural viability \(\kappa\) measures the remaining capacity of the structure
to sustain its own existence. It is a strictly non-increasing quantity and cannot
be restored, replenished, or optimized.

When \(\kappa > 0\), the living structural state exists and may give rise to
physical projections. When \(\kappa = 0\), structural evolution terminates. No
future structural states exist beyond this point.

Structural termination is not a physical event. It does not correspond to
destruction, explosion, or divergence in physical quantities. Rather, it marks
the boundary beyond which physical description is no longer defined, because the
structural substrate required for projection no longer exists.

% -----------------------------
\section{Physics as a Projection Domain}

In Flexion Physics, physical reality is not identified with the underlying
structural ontology. Instead, physics is defined as a projection domain: a
derived descriptive layer that becomes meaningful only when a living structural
state exists and remains viable.

This distinction separates structural existence from physical description and
prevents the conflation of physical laws with fundamental ontology.

\subsection{Projection Operator \(P_{\mathrm{phys}}\)}

Physical observables arise through a projection operator
\[
\text{Physics} = P_{\mathrm{phys}}(X),
\]
which maps the living structural state
\[
X = (\Delta, \Phi, M, \kappa)
\]
into a domain of physical quantities, relations, and laws. The operator
\(P_{\mathrm{phys}}\) does not preserve the full structure of \(X\); it produces
only those features that admit physical interpretation.

The projection operator is non-invertible. Multiple structural configurations
may correspond to the same physical description, and no physical measurement can
reconstruct the underlying structural state uniquely. Physics therefore carries
less information than structure and cannot serve as a complete ontological
foundation.

\subsection{Non-Identity of Structural and Physical Variables}

Structural variables are not physical variables. In particular,
structural deformation \(\Delta\) is not matter,
structural energy \(\Phi\) is not physical energy,
structural memory \(M\) is not entropy,
and structural viability \(\kappa\) is not physical existence.

Physical quantities emerge as modes of interpretation of structural relations,
not as direct representations of structural components. Any attempt to identify
structural variables with physical entities leads to category errors and
conceptual contradictions.

This non-identity ensures that Flexion Physics does not compete with existing
physical theories. Physical models operate entirely within the projection
domain and remain valid insofar as the projection itself remains well-defined.

\subsection{Conditions of Physical Validity}

The existence of physics is conditional. Physical description is meaningful if
and only if structural viability is positive:
\[
\kappa > 0 \quad \Longleftrightarrow \quad \text{Physics exists}.
\]
When \(\kappa = 0\), the projection operator \(P_{\mathrm{phys}}\) ceases to be
defined. Physics does not fail dynamically; it becomes ontologically undefined.

This condition establishes a strict domain of validity for all physical laws.
Physical singularities, divergences, or breakdowns do not indicate the presence
of infinite physical quantities, but rather the loss of the structural substrate
required for projection.

Physics is therefore contingent, not fundamental. It exists only as long as the
living structural state sustains the conditions necessary for its projection.

% -----------------------------
\section{Origin of Physical Time}

The origin of physical time is one of the central problems of physics. Classical
and modern theories typically assume time as a fundamental parameter or
coordinate, while explaining its arrow through thermodynamic or statistical
arguments. Flexion Physics adopts a different approach: physical time is not
fundamental and does not exist independently of structural processes.

\subsection{Time as a Projection of \(M\) and \(\kappa\)}

Structural time arises from the irreversible accumulation of structural memory
\(M\). The ordering induced by the monotonic growth of \(M\) defines a sequence
of structural states that cannot be reversed. This ordering constitutes
structural time.

Physical time exists only as a projection of this ordering and only while
structural viability remains positive. The projection of structural time into
the physical domain depends jointly on memory and viability. When \(\kappa > 0\),
structural evolution may be mapped to a physical time parameter. When
\(\kappa = 0\), no physical time parameter can be defined.

Thus, physical time is neither absolute nor fundamental. It is a derived
quantity whose existence depends on the continued viability of the underlying
structure.

\subsection{Arrow of Time}

The arrow of physical time is inherited directly from structural irreversibility.
Because structural memory cannot be erased or decreased, the projected physical
time parameter necessarily possesses a preferred direction.

This arrow does not depend on probabilistic assumptions, entropy maximization,
or coarse-graining. It is a structural consequence of memory accumulation and
exists even in the absence of thermodynamic considerations.

As a result, the irreversibility observed in physical processes reflects a
deeper structural irreversibility rather than an emergent statistical property
of physical systems.

\subsection{Breakdown of Time at Collapse}

When structural viability reaches zero, structural evolution terminates. Beyond
this point, no further structural states exist, and the projection of time
ceases.

The disappearance of physical time at collapse does not correspond to a
singularity or divergence in temporal quantities. Instead, it marks the loss of
the conditions under which time can be defined at all.

There is no meaningful notion of physical time beyond structural collapse.
Questions about temporal continuation, evolution, or dynamics past this boundary
are therefore ill-posed. Collapse represents the terminal boundary of time, not
an extreme moment within it.

% -----------------------------
\section{Energy, Forces, and Dynamics}

In classical physics, energy and forces are treated as fundamental quantities
governing the dynamics of physical systems. Flexion Physics reinterprets these
concepts as projections of deeper structural relations. Energy and forces do not
exist at the structural level; they arise only within the physical projection
domain.

\subsection{Structural Energy and Physical Energy}

Structural energy \(\Phi\) represents internal tension within the living
structural state. It is not a physical quantity and does not correspond directly
to any measurable form of physical energy.

Physical energy emerges as a projection of structural tension. Different regimes
of structural energy may give rise to different physical energy modes, such as
kinetic, potential, or field energy, depending on the structure of the
projection. These modes do not exhaust the meaning of \(\Phi\); they are
interpretations constrained by the projection operator \(P_{\mathrm{phys}}\).

Because physical energy is derived, its conservation reflects the preservation
of underlying structural invariants rather than a fundamental physical law.

\subsection{Forces as Structural Gradients}

Forces in the physical domain arise as projected gradients of structural
relations. In particular, variations in structural deformation \(\Delta\) and
structural energy \(\Phi\) give rise to effective physical forces under
projection.

These forces do not exist independently at the structural level. They are
context-dependent manifestations of how structural tensions are interpreted
within the physical domain. As such, the same structural configuration may give
rise to different force descriptions under different projection regimes.

This interpretation removes the need to treat forces as primitive entities and
allows them to be understood as derived effects of structural geometry.

\subsection{Conservation Laws as Structural Invariants}

Conservation laws play a central role in physical theory. In Flexion Physics,
such laws are understood as reflections of invariant properties of the living
structural state under projection.

Conservation of energy, momentum, or other physical quantities does not imply
that these quantities are fundamentally conserved at the structural level.
Rather, conservation emerges because the projection preserves certain
relationships among structural variables as long as structural viability
remains positive.

When the conditions required for projection break down, conservation laws lose
their meaning. This does not constitute a violation of physical law, but rather
the disappearance of the domain in which such laws apply.

% -----------------------------
\section{Singularities as Projection Failure}

Physical singularities occupy a central and unresolved position in modern
physics. In classical and relativistic theories they appear as divergences of
curvature, density, or energy, while in quantum theory they signal the breakdown
of perturbative descriptions. Flexion Physics proposes a structural resolution
of singularities that does not invoke infinite physical quantities.

\subsection{Structural Interpretation of Singularities}

Within the Flexion Framework, singular behavior in physical models does not
correspond to a singularity in structure. Instead, it indicates a failure of the
projection operator \(P_{\mathrm{phys}}\) as structural viability approaches
zero.

As \(\kappa \to 0\), the conditions required to interpret structural relations as
physical quantities are progressively lost. Physical variables may appear to
diverge, oscillate, or become undefined, but these effects reflect the collapse
of the projection domain rather than pathological behavior of the structure
itself.

Singularities are therefore not events within physics. They are boundaries of
physics.

\subsection{No Physical Infinities}

Flexion Physics asserts that no physical infinity exists as an ontological
entity. Apparent infinities arise only within mathematical formalisms that are
applied beyond the domain where physical projection remains valid.

When projection fails, physical quantities lose their interpretability before
they become infinite. Divergence signals the exhaustion of the projection, not
the existence of unbounded physical values.

This perspective removes the need to interpret singularities as physically real
objects or states. Instead, they mark the limit beyond which physical description
ceases to apply.

\subsection{Implications for Relativity and Quantum Theory}

In general relativity, spacetime singularities such as those associated with
black holes represent points where the geometric description of spacetime
breaks down. In quantum theory, divergences signal the limits of perturbative
expansion and measurement.

Flexion Physics does not modify these theories or propose alternative equations.
Rather, it provides a common structural explanation for their breakdown: both
encounter regimes where projection from structure to physics is no longer
defined.

From this perspective, unifying physics does not require eliminating
singularities within physical theory. It requires recognizing singularities as
indicators of projection failure and respecting the structural boundary they
represent.

% -----------------------------
\section{Relation to Existing Physical Theories}

Flexion Physics is not proposed as an alternative to established physical
theories. It does not introduce new equations of motion, does not modify known
laws, and does not attempt to subsume existing models under a unified physical
formalism. Its role is ontological rather than predictive.

\subsection{Compatibility, Not Replacement}

All established physical theories operate entirely within the physical
projection domain. Classical mechanics, electrodynamics, quantum theory, and
general relativity remain valid descriptions of physical phenomena insofar as
the projection from structure to physics remains well-defined.

Flexion Physics neither contradicts nor supersedes these theories. Instead, it
clarifies the conditions under which they apply and the reason they fail at their
respective limits. Where physical theories encounter singularities, divergences,
or interpretational paradoxes, Flexion Physics identifies these as boundaries of
projection rather than failures of the theories themselves.

This compatibility ensures that Flexion Physics does not compete with existing
models and does not require experimental validation in the traditional sense.

\subsection{Why Reduction Is Impossible}

A common expectation in foundational physics is that deeper theories should
reduce higher-level descriptions to more fundamental physical entities. Flexion
Physics explicitly rejects this expectation.

Structure is not physical and cannot be reduced to particles, fields, spacetime,
or information. Physical quantities arise only as projections of structure and
therefore cannot serve as its constituents. Any attempt to reduce structural
variables to physical ones necessarily inverts the projection and results in
conceptual inconsistency.

For this reason, no physical theory, regardless of its mathematical
sophistication, can fully explain the origin of time, irreversibility,
singularities, or the existence of physics itself. These questions belong to the
structural domain addressed by the Flexion Framework and its physical projection.

Flexion Physics thus occupies a meta-theoretical position: it explains why
physical theories work when they do, and why they inevitably fail when they
reach the limits of their projection domain.

% -----------------------------

\section{Theorems}

This section states the core formal consequences of Flexion Physics. The theorems
do not introduce new assumptions beyond the structural ontology defined earlier.
They articulate necessary conditions for the existence of physics, time, and
singular behavior as consequences of projection from structure.

\begin{theorem}[Existence of Physics]
Physical observables exist if and only if structural viability is positive:
\[
\kappa > 0 \;\Longleftrightarrow\; \text{Physics exists}.
\]
\end{theorem}

\begin{proof}
Physical observables arise exclusively through the projection operator
\(P_{\mathrm{phys}}\). This operator is defined only while the living structural
state exists and remains viable. When \(\kappa > 0\), projection is possible and
physical quantities may be defined. When \(\kappa = 0\), structural evolution
terminates and no projection domain exists. Hence, physics exists if and only if
\(\kappa > 0\).
\end{proof}

\begin{theorem}[Non-Existence of Physical Singularities]
No physical singularity exists as an ontological entity. All apparent
singularities correspond to failure of the projection operator
\(P_{\mathrm{phys}}\).
\end{theorem}

\begin{proof}
As structural viability approaches zero, the conditions required for projection
into physical quantities degrade. Mathematical divergences appear only when
formal physical descriptions are extrapolated beyond this domain. Since
projection ceases before infinite physical values acquire meaning, no physical
infinity exists. Apparent singularities therefore indicate projection failure,
not physical divergence.
\end{proof}

\begin{theorem}[Irreversibility of Physical Time]
Physical time is irreversible.
\end{theorem}

\begin{proof}
Physical time is a projection of structural ordering induced by accumulated
structural memory \(M\). Because \(M\) is irreversible, the ordering of structural
states cannot be reversed. The projected physical time parameter inherits this
irreversibility directly, independent of statistical or thermodynamic
considerations.
\end{proof}

\begin{theorem}[Termination of Physical Time]
Physical time ceases to exist at structural collapse.
\end{theorem}

\begin{proof}
When \(\kappa = 0\), no further structural states exist and projection terminates.
Since physical time exists only as a projection of structural evolution, it
cannot be defined beyond this boundary. Therefore, physical time terminates at
structural collapse.
\end{proof}

Together, these theorems establish that physics is contingent, irreversible, and
structurally bounded. Physical laws do not fail at their extremes; rather, the
domain in which they are defined ceases to exist.

% -----------------------------
\section{Discussion}

Flexion Physics V1.2 occupies a position outside the traditional landscape of
physical theory. It neither extends existing models nor proposes alternative
dynamical laws. Instead, it addresses a class of questions that physical theory
itself is structurally incapable of resolving: the origin of physical time, the
meaning of singularities, and the conditions under which physics exists at all.

A central implication of this framework is the recognition that many long-standing
problems in physics arise from category errors. Treating time, energy, or
spacetime as ontologically fundamental leads inevitably to paradoxes when these
concepts are pushed to their limits. By relocating these quantities to the
projection domain, Flexion Physics dissolves such paradoxes without modifying the
successful predictive machinery of physical theory.

The interpretation of singularities as projection failures rather than physical
objects provides a unified perspective across classical and quantum regimes.
Instead of attempting to regularize infinities or quantize spacetime itself,
Flexion Physics asserts that physical description loses meaning before such
extremes are reached. This shifts the focus from technical resolution to
ontological clarity.

Flexion Physics also reframes irreversibility. The arrow of time is not an
emergent statistical artifact but a necessary consequence of structural memory.
This perspective explains why irreversibility persists across scales and
contexts, even where thermodynamic arguments are insufficient or ambiguous.

Importantly, the theory imposes strict limits on its own claims. It does not
predict new physical phenomena, does not propose testable deviations from known
laws, and does not assert empirical superiority over existing theories. Its value
lies in clarifying what physical theories describe, why they work, and why they
inevitably encounter boundaries.

In this sense, Flexion Physics should be understood as a boundary theory. It
defines the outer limits of physical description and provides a coherent
framework for understanding why those limits exist, without attempting to
transcend them within physics itself.

% -----------------------------
\section{Conclusion}

Flexion Physics V1.2 has presented a structural interpretation of physics in which
physical reality is understood as a projection of a deeper living structural
state. The theory does not alter existing physical laws or models, but instead
clarifies the ontological conditions under which such laws are meaningful.

By introducing a strict separation between structure and physics, Flexion
Physics explains the origin of physical time, energy, forces, and conservation
laws as derived phenomena. Structural memory provides the foundation for the
arrow of time, while structural viability defines the domain in which physical
description exists at all.

A central result of the theory is the reinterpretation of singularities.
Physical singularities are not infinite physical states, but boundaries where
projection from structure to physics ceases to be defined. In this view, the
breakdown of physical theories at extreme regimes reflects the exhaustion of
their domain of applicability rather than a failure of the theories themselves.

Flexion Physics also establishes irreversibility as a structural necessity rather
than a statistical artifact. Physical time inherits its direction from the
irreversible accumulation of structural memory and terminates naturally at
structural collapse.

The scope of Flexion Physics is deliberately limited. It makes no empirical
predictions, proposes no modifications to established theories, and does not
seek experimental confirmation. Its contribution is conceptual: it provides a
coherent ontological framework that explains why physics works, why it fails at
its boundaries, and why those boundaries cannot be removed from within physics
itself.

In doing so, Flexion Physics V1.2 defines a stable foundation for understanding
physical reality as contingent, irreversible, and structurally bounded.

% -----------------------------
\begin{thebibliography}{9}
\bibitem{framework}
Flexion Framework, DOI (10.5281/zenodo.17860282).

\bibitem{time}
Flexion Time Theory, DOI (10.5281/zenodo.17668314).

\bibitem{collapse}
Flexion Collapse Theory, DOI (10.5281/zenodo.17726503).
\end{thebibliography}

\end{document}
