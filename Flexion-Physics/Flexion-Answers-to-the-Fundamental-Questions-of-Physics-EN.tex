\documentclass[11pt]{article}

% -----------------------------
% Packages
% -----------------------------
\usepackage[a4paper,margin=1in]{geometry}
\usepackage{hyperref}
\usepackage{enumitem}
\usepackage{authblk}

% -----------------------------
% Hyperref setup
% -----------------------------
\hypersetup{
  colorlinks=true,
  linkcolor=blue,
  urlcolor=blue
}

% -----------------------------
% Title and Author
% -----------------------------
\title{\textbf{Flexion: Answers to the Fundamental Questions of Physics}}
\author{Maryan Bogdanov}
\affil{Independent Researcher}
\date{\today}

\begin{document}
\maketitle

% -----------------------------
% Abstract
% -----------------------------
\begin{abstract}
This document presents a structural perspective on the fundamental questions of
physics. Rather than proposing new physical laws or equations, Flexion interprets
physics itself as a projection of a deeper living structure underlying reality.
The text is organized as a set of concise questions and answers addressing the
origin of time, irreversibility, energy, singularities, quantum uncertainty,
cosmology, and the limits of physical theories. Its purpose is not to replace
physics, but to clarify its domain of validity and its unavoidable boundaries.
\end{abstract}

% -----------------------------
\section{Introduction}

Modern physics is one of the most successful ways of describing nature.
It allows us to predict phenomena with high precision and to create technologies
that profoundly shape the world. However, despite its achievements, physics
leaves a number of fundamental questions unanswered.

Why do physical laws exist at all?  
Where do time and irreversibility come from?  
Why do singularities appear, and why do physical theories fail at their extreme
limits?  
Why has a complete “theory of everything” remained so elusive?

Flexion approaches these questions from a structural perspective.
It is not a new physical theory and does not compete with existing models.
Instead, Flexion treats physics itself as a projection of a deeper, living
structure underlying reality.

This document presents concise answers to the fundamental questions of physics
from the viewpoint of Flexion. Its purpose is not to replace physical theories,
but to clarify what they describe, why they work within certain limits, and why
some questions cannot be meaningfully formulated from within physics alone.

% -----------------------------
\section{Questions and Answers}

% =====================================================
% Questions 1–15 (FULL, UNSHORTENED)
% =====================================================

\subsection*{1. What is physics, really?}
\textbf{Short answer:} Physics is not the foundation of reality, but a way of describing it.

Physics does not describe what exists in itself.
It describes what becomes observable and measurable under certain conditions.
Physical laws are rules for the consistent interpretation of observed reality,
not properties of the underlying structure itself.
Within Flexion, physics is understood as a projection of a deeper living
structure, not as its ontological basis.

\subsection*{2. Where do space and time come from?}
\textbf{Short answer:} Space and time are not fundamental entities.

Space arises as a physical interpretation of structural deformations.
Time arises from the irreversible accumulation of structural memory.
Without a living structure, neither space nor time exists—only the absence of
physical description.
Physical spacetime exists solely as the result of projection.

\subsection*{3. Why does time have an arrow?}
\textbf{Short answer:} Because memory is irreversible.

The arrow of time is not a consequence of entropy or probabilistic processes.
It arises from the fact that structural memory cannot be erased or reduced.
Physical time inherits this irreversibility through projection, which is why the
past is fundamentally different from the future.

\subsection*{4. What is energy?}
\textbf{Short answer:} Energy is a projection of structural tension.

Energy is not a fundamental entity.
It appears as a physical interpretation of structural energy—internal tension
within the living structure.
Different forms of energy in physics correspond to different modes of projection,
not to different kinds of fundamental reality.

\subsection*{5. What are forces?}
\textbf{Short answer:} Forces are effects, not entities.

Physical forces arise as the result of projecting structural gradients.
They do not exist at the structural level and are not primitive objects.
Forces are the way physics describes change of state within its projected view
of reality.

\subsection*{6. Why do conservation laws work?}
\textbf{Short answer:} Because projection preserves structural invariants.

Conservation laws are not fundamental laws of nature.
They reflect the fact that, during physical projection, certain relationships
between structural quantities remain preserved.
As long as the projection remains well-defined, physics observes these preserved
relationships as conservation of energy, momentum, and other quantities.

\subsection*{7. What are singularities?}
\textbf{Short answer:} Singularities are not infinities, but boundaries of physics.

Singularities do not arise because physical quantities become infinite.
They appear when physical projection ceases to be valid.
Physics does not encounter “infinity”; it loses applicability.
Singularities therefore mark the limits of physical description, not extreme
states of reality.

\subsection*{8. What happened before the Big Bang?}
\textbf{Short answer:} The question is incorrectly formulated.

The notion of “before” presupposes the existence of time.
However, time is a physical projection that exists only in the presence of a
living structure.
Outside this structure, physical time does not exist, and therefore the concept
of “before” has no meaningful interpretation.

\subsection*{9. Can quantum mechanics and gravity be unified?}
\textbf{Short answer:} Not within physics itself.

Quantum mechanics and gravity operate in different regimes of physical
projection.
Their incompatibility is not a technical problem, but a structural one.
Unification is possible only at the level of structure, not in the form of yet
another physical theory with new equations.

\subsection*{10. Why does physics exist at all?}
\textbf{Short answer:} Because a living structure exists.

Physics is not an inevitable property of reality.
It exists only as long as the living structure remains viable.
When structural viability is exhausted, physics does not collapse or undergo a
catastrophe—it simply ceases to be definable.

\subsection*{11. Can time be reversed?}
\textbf{Short answer:} No, not in principle.

Any reversibility in physics is local and conditional.
At the structural level, memory is irreversible, which makes complete
reversibility impossible even in theory.
Physical equations may be formally time-reversible, but they operate on top of
an irreversible structure and cannot undo its progression.

\subsection*{12. Does the observer play a fundamental role in physics?}
\textbf{Short answer:} No. The observer is not fundamental, but not irrelevant.

The observer does not create reality and does not determine the laws of physics.
However, every observation is an act of physical projection.
Measurement does not modify structure, but it participates in forming the
physical description.
In this sense, the observer belongs not to structure itself, but to the level of
physical interpretation.

\subsection*{13. Is consciousness directly related to physics?}
\textbf{Short answer:} No. Consciousness is not a physical object.

Consciousness cannot be reduced to particles, fields, or information, because
physics itself is a projection.
Flexion does not introduce consciousness as a separate entity and does not offer
a mechanical explanation of it.
It only states that consciousness does not belong to the physical level of
description and cannot be fully explained within physics.

\subsection*{14. What are black holes, really?}
\textbf{Short answer:} Black holes are boundaries of physical projection.

In standard physics, black holes are described as regions of extreme spacetime
curvature.
From a structural perspective, this means that physical projection in these
regions becomes increasingly ill-defined.
The event horizon is not a physical surface, but a boundary of valid physical
interpretation.

\subsection*{15. What happens inside a black hole?}
\textbf{Short answer:} Nothing physical.

Asking what happens “inside” presupposes the existence of space and time.
However, under structural collapse, the physical projections of space and time
lose definition.
Therefore, inside a black hole there are no physical processes—only the limit of
physical description.

% =====================================================
% Questions 16–30 (FULL, UNSHORTENED)
% =====================================================

\subsection*{16. What is dark matter?}
\textbf{Short answer:} Dark matter is an effect of incomplete physical projection of structure.

Dark matter manifests through gravitational effects but is not directly
observable.
Within Flexion, this indicates that certain structural deformations project into
physics as gravity without forming observable matter.
Dark matter does not have to consist of particles; it may be a manifestation of
structural content that lacks a complete physical representation.

\subsection*{17. What is dark energy?}
\textbf{Short answer:} Dark energy is a projection of global structural tension.

The accelerated expansion of the universe indicates that the underlying structure
is in a state of tension.
Physics interprets this as dark energy, but it does not correspond to a new form
of energy.
Instead, it reflects global structural dynamics that cannot be localized into
particles or fields.

\subsection*{18. Why can we not directly detect dark matter and dark energy?}
\textbf{Short answer:} Because they are not required to be fully physical.

If an observed effect results from partial or distorted projection, attempts to
detect it as an ordinary physical entity may fail.
Flexion does not exclude the possibility of dark matter particles, but it
explains why their absence is not a conceptual problem for understanding the
observed phenomena.

\subsection*{19. Is the universe finite or infinite?}
\textbf{Short answer:} The question is physically ill-posed.

Finiteness and infinity are properties of mathematical models, not of structure
itself.
The physical universe exists only as a projection of a living structure and
inherits its limitations.
Physical infinity is impossible because a projection cannot be infinite.

\subsection*{20. Why does the physical universe exist at all?}
\textbf{Short answer:} Because a viable structure exists.

The universe is not a fundamental entity.
It exists only as a physical projection of a living structure while that
structure remains viable.
When structural viability is exhausted, what disappears is not “the universe”
but the possibility of its physical description.

\subsection*{21. Why are physical laws the same throughout the universe?}
\textbf{Short answer:} Because physics is a projection of a single structure.

Physics assumes the universality of laws but does not explain its origin.
Within Flexion, this universality follows from the fact that the physical world is
a single projection of a living structure.
Physical laws are the same everywhere not because they “spread” across the
universe, but because the projection is the same wherever it is defined.

\subsection*{22. Why do physical constants have exactly these values?}
\textbf{Short answer:} Because they are parameters of projection, not properties of reality.

Physical constants are not chosen, tuned, or selected.
They reflect admissible regimes in which structure can be interpreted as
physical reality.
This removes the need for explanations based on fine-tuning, chance, or the
existence of a multiverse.

\subsection*{23. Why is quantum mechanics probabilistic?}
\textbf{Short answer:} Because physical projection is irreversible and non-invertible.

Probability in quantum mechanics is not a fundamental property of reality.
It arises from the loss of structural information during projection into
physical description.
Physics cannot access structure directly and therefore must operate using
probabilistic distributions.

\subsection*{24. What does quantum uncertainty really mean?}
\textbf{Short answer:} It is not uncertainty of reality, but uncertainty of description.

Quantum uncertainty reflects the impossibility of reconstructing the structural
state from physical data.
It does not imply that objects are “smeared” or do not exist.
Rather, it indicates that physics is not a fundamental level of description and
necessarily loses information during projection.

\subsection*{25. Why is measurement a special process in quantum mechanics?}
\textbf{Short answer:} Because measurement is an act of projection, not an ordinary physical interaction.

Measurement is not a dynamical process occurring within physics.
It represents a transition from structural state to physical description.
The so-called “collapse of the wave function” is not a physical event, but a
change in the mode of interpretation of structure.

\subsection*{26. Why does the macroscopic world appear classical while the microscopic world appears quantum?}
\textbf{Short answer:} Because these are different stable regimes of physical projection.

Classical behavior is not a fundamental property of reality.
It emerges where physical projection stabilizes and loses sensitivity to fine
structural details.
Quantum behavior appears in regimes where this sensitivity is preserved.
The distinction between micro and macro reflects properties of projection, not a
change in underlying reality.

\subsection*{27. What is information, really?}
\textbf{Short answer:} Information is not a physical entity.

Information is a trace of structural memory within physical description.
It does not exist independently and cannot serve as a foundation of reality.
Physics works with information because it lacks direct access to structure and
operates on its projected traces.

\subsection*{28. Why is the universe irreversible even in principle?}
\textbf{Short answer:} Because structure itself is irreversible.

Even if physical equations are formally reversible, they operate on top of
irreversible structural memory.
Physical irreversibility is not an approximation or a statistical artifact—it is
inherited from structure and is therefore fundamental.

\subsection*{29. Why is it impossible to “rewind” the universe?}
\textbf{Short answer:} Because physical time is not the carrier of reality.

Rewinding presupposes the ability to return to a previous state.
However, physical time is only a projection of structural ordering.
Once the structure has changed, physical description cannot be rolled back,
because the basis of projection has already been altered.

\subsection*{30. Is Flexion a Theory of Everything?}
\textbf{Short answer:} Yes—but not in the physical sense.

Flexion is not a physical Theory of Everything and does not propose unified
equations describing all interactions.
It is a structural theory that describes the origin of physics, time, laws, and
their limits.
In this sense, Flexion encompasses everything, but not as a physical model—
rather as an ontological framework explaining why no physical theory can be a
Theory of Everything.

% -----------------------------
\section{Conclusion}

In this document, Flexion has been presented as a structural perspective on the
fundamental questions of physics. Rather than introducing new physical entities
or equations, it shows that physical laws, quantities, and concepts arise as
projections of a deeper living structure underlying reality.

This approach provides a coherent way to understand the origin of time,
irreversibility, conservation laws, quantum uncertainty, singularities, and
cosmological limits. Many problems traditionally regarded as “unsolvable” in
physics turn out not to be physical mysteries, but consequences of exceeding the
domain in which physical description remains meaningful.

Flexion does not abolish physics and does not replace existing theories. It
clarifies their status, their domain of applicability, and their fundamental
limitations. Physics remains a powerful and indispensable tool for describing
nature, but it no longer claims to be the ultimate foundation of reality.

In this sense, Flexion is not a physical Theory of Everything. It is a structural
framework that explains why physical theories work, why they inevitably have
limits, and why certain questions cannot be meaningfully formulated within
physics itself. By making these boundaries explicit, Flexion enables a more
honest and deeper understanding of physics as it truly is.

\end{document}
