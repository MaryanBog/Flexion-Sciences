\documentclass[11pt]{article}

% Encoding and language
\usepackage[utf8]{inputenc}
\usepackage[T1]{fontenc}
\usepackage[english]{babel}

% Math
\usepackage{amsmath, amssymb, amsfonts}

% Page layout (arXiv-like)
\usepackage{geometry}
\geometry{margin=1in}

% Better formatting
\usepackage{setspace}
\onehalfspacing

% Links
\usepackage{hyperref}
\hypersetup{
    colorlinks = true,
    linkcolor = blue,
    citecolor = blue,
    urlcolor = blue
}

% Section spacing similar to arXiv
\usepackage{titlesec}
\titleformat{\section}{\large\bfseries}{\thesection}{0.8em}{}
\titleformat{\subsection}{\normalsize\bfseries}{\thesubsection}{0.6em}{}

% Document begins
\begin{document}

% Title
\title{\textbf{Flexionization as a Universal Model of Immune Dynamics}\\
\vspace{0.3em}
\large Flexion-Immune-Model Version 1.1}

% Author
\author{
Maryan Bogdanov \\
\small Independent Researcher \\
\small \texttt{m7823445@gmail.com}
}

% Date
\date{2025}

% Render title
\maketitle

\begin{abstract}
This paper presents a conceptual model of immune dynamics based on the Flexionization theory. 
It shows that the key structural elements of Flexionization --- synthetic mass $Q_p$, reference structural mass $Q_F$, 
structural deviation $\Delta$, and the equilibrium indicator $\mathrm{FXI}$ --- naturally correspond to biological quantities: 
the actual physiological state of the organism, the healthy reference state, the degree of pathological deviation, 
and an integrated indicator of immune balance.

The immune response is interpreted as a corrective operator $E$ that strives to return the system to structural symmetry 
$\mathrm{FXI} = 1$. Within this interpretation, a dynamic equation of the immune response is derived, describing either 
the organism's restoration of health or the loss of stability when the compensatory reaction is insufficient.

The proposed model is universal, mathematically rigorous, and applicable to various immune scenarios --- viral infections, 
oncological processes, autoimmune disorders, and immunodeficiency states. This work opens the possibility of establishing 
a unified formal platform for studying immune dynamics.
\end{abstract}

\newpage

\section{Introduction}

The study of the immune system is traditionally associated with significant complexity due to its nonlinearity, 
multi-level organization, and deep dependencies between cellular, biochemical, and systemic processes. 
Most existing models of the immune response are based on biological mechanisms, empirical equations, or statistical 
approximations, which limits the ability to derive universal laws of immune dynamics and to predict system behavior 
across diverse pathological conditions.

At its core, immunity is a regulatory mechanism with feedback: it measures the degree of deviation of the organism 
from a reference healthy state and forms a corrective response aimed at restoring equilibrium. However, the formal 
mathematical description of such regulation remains fragmented and is not unified into a coherent theoretical structure.

The Flexionization theory provides a universal, axiomatically constructed model of dynamic structural equilibrium, 
originally developed to describe systems where maintaining balance between the actual and reference state is essential. 
The key elements of the theory --- synthetic mass $Q_p$, structural mass $Q_F$, deviation $\Delta$, and the equilibrium 
indicator $\mathrm{FXI}$ --- can be naturally interpreted in a biological context, where the organism is viewed as a 
dynamic system striving for structural symmetry ($\mathrm{FXI} = 1$), corresponding to a healthy state.

The purpose of this work is to demonstrate that Flexionization can serve as a universal mathematical foundation for 
modeling immune dynamics, providing formal rules, operator mechanisms, stability criteria, and a generalized description 
of the organism's response to pathogenic disturbances.

\section{The Immune System as a Dynamic Regulatory Structure}

The immune system is a complex mechanism responsible for maintaining the internal balance of the organism. 
Its primary function is to detect deviations from normal physiological conditions and to form a corrective 
response aimed at restoring stability.

From the standpoint of dynamical systems theory, immunity can be viewed as a feedback regulator. 
The system continuously measures the current state of the organism --- the presence of pathogens, 
inflammatory markers, cellular damage --- and compares it with a reference healthy state. Based on this 
comparison, a corrective action is generated, varying in strength and nature: from activation of innate 
mechanisms to complex adaptive immune responses.

This perspective aligns well with control theory:
\begin{itemize}
    \item there exists a target (reference) state of the organism,
    \item there exists a measurable deviation from this state,
    \item there exists a regulator attempting to reduce this deviation.
\end{itemize}

However, existing biological models of immunity often focus on specific mechanisms or localized 
interactions between cells, making them highly specialized and difficult to generalize. As a result, 
there is no unified formal structure capable of describing immune dynamics as a whole, independently 
of specific diseases or biochemical details.

This creates the need for an abstract regulatory model that can capture the general principles of 
immune function at the level of system dynamics. Flexionization can serve as such a model, as its 
axiomatic framework fully corresponds to the requirements of a formal description of immune equilibrium.

\section{Overview of the Flexionization Structure}

The Flexionization theory describes dynamic equilibrium in systems where it is necessary to maintain 
structural balance between the actual state and the reference state. The model is built on an axiomatic 
foundation and defines a formal set of quantities that describe the structure of the system at any moment 
in time.

Below are the key elements of the theory, which will later be interpreted in a biological context.

\subsection{Key Quantities $Q_p$, $Q_F$, and $\Delta$}

\textbf{Synthetic mass $Q_p$} represents the actual structural mass of the system.  
It reflects the current state of the system, aggregating all significant parameters into a single value.

\textbf{Structural mass $Q_F$} represents the ideal (reference) state of the system against which the actual 
state is compared.

\textbf{Deviation $\Delta = Q_p - Q_F$} quantifies the structural difference between the actual and reference 
states.  
A positive value indicates excess, a negative value indicates deficit, and zero represents perfect balance.

This triad $(Q_p, Q_F, \Delta)$ allows any imbalance within the system to be expressed formally.

\subsection{Equilibrium Indicator $\mathrm{FXI}$}

The equilibrium indicator $\mathrm{FXI}$ is a monotonic mapping of the system's state that measures the 
degree of deviation from structural symmetry.

\begin{itemize}
    \item $\mathrm{FXI} > 1$ --- expanded structural state,
    \item $\mathrm{FXI} < 1$ --- compressed structural state,
    \item $\mathrm{FXI} = 1$ --- perfect structural symmetry (equilibrium).
\end{itemize}

$\mathrm{FXI}$ serves as an integral metric that reflects the current structural quality of the system.

\subsection{Corrective Operator $E$}

The corrective operator $E$ determines how the system adjusts from one state to the next.  
It specifies how deviation from equilibrium should be corrected in the upcoming step.

If $\mathrm{FXI}$ measures deviation, then $E(\mathrm{FXI})$ gives the target equilibrium value for the next moment.

The operator $E$ satisfies the following:

\begin{itemize}
    \item it is defined for all admissible states,
    \item it is monotonic,
    \item it is bounded,
    \item it tends to drive the system toward structural symmetry ($\mathrm{FXI} = 1$).
\end{itemize}

In a biological context, the corrective operator corresponds to the immune response.

\section{Mapping Immunity onto the Flexionization Structure}

We now examine how the biological elements of the immune system naturally correspond to the structural 
quantities defined in the Flexionization theory. This mapping enables immune dynamics to be formalized 
using an axiomatic framework originally developed for structural regulatory systems.

\subsection{The Organism as a System State $S$}

In Flexionization, the state of a system is represented as the tuple:
\[
S = (Q_p, Q_F, \Delta, q, W, U)
\]
where $q$, $W$, and $U$ represent additional structural parameters.

In the biological context:
\begin{itemize}
    \item $Q_p$ --- the actual physiological state of the organism (e.g., pathogen load, inflammation level, immune cell activity),
    \item $Q_F$ --- the healthy reference state of the organism,
    \item $q$ --- a set of biomarkers (cellular and biochemical indicators),
    \item $W$ --- the relative weights of these indicators,
    \item $U$ --- internal parameters of the organism (genetics, baseline immune tone, individual characteristics).
\end{itemize}

Thus, the organism can be viewed as a formal system whose state at any moment corresponds precisely 
to a Flexionization state $S$.

\subsection{Disease as Structural Deviation $\Delta$}

Pathological processes cause the actual structural mass $Q_p$ to diverge from the reference mass $Q_F$.  
The degree of mismatch is
\[
\Delta = Q_p - Q_F .
\]

Biologically, this corresponds to:
\begin{itemize}
    \item increased viral load,
    \item uncontrolled tumor cell proliferation,
    \item dysregulated inflammation,
    \item immunodeficiency.
\end{itemize}

If $\Delta > 0$, the system is displaced toward pathological excess  
(e.g., rapid viral replication, tumor expansion).

If $\Delta < 0$, the system exhibits deficit or insufficiency  
(e.g., immune suppression, tissue damage, exhaustion of reserves).

Thus, $\Delta$ provides a strict quantitative measure of disease.

\subsection{The Immune Response as the Corrective Operator $E$}

The corrective operator $E$ determines the target value of the equilibrium indicator $\mathrm{FXI}$ for the next state.

In biological terms, $E$ corresponds to the \textbf{immune response}, which includes:
\begin{itemize}
    \item activation of innate immunity,
    \item adaptive immune response,
    \item antibody production,
    \item cytotoxic T-cell activity,
    \item inflammatory mechanisms,
    \item restoration of homeostasis.
\end{itemize}

In essence, $E$ is the biological correction mechanism that attempts to reduce $\Delta$ and restore equilibrium 
($\mathrm{FXI} = 1$).

\subsection{Health as Structural Symmetry ($\mathrm{FXI} = 1$)}

In Flexionization, the condition $\mathrm{FXI} = 1$ represents perfect structural symmetry.

In biological terms, this corresponds to:
\begin{itemize}
    \item absence of pathogens,
    \item normal immune cell levels,
    \item minimal inflammatory activity,
    \item stable biomarkers.
\end{itemize}

Thus, $\mathrm{FXI} = 1$ corresponds to a state of \textbf{health}, and deviations from this value quantify 
the degree of pathology.

This structural mapping provides a universal mathematical basis for describing immune dynamics.

\section{Dynamics of the Immune Response in Flexionization Terms}

The Flexionization theory defines a formal rule for how the system transitions from one state to another 
under the action of the corrective operator $E$. In the biological context, this rule describes how the 
organism responds to pathological deviation and attempts to return to a healthy state.

\subsection{Core Dynamic Equation for $\Delta$}

In Flexionization, the evolution of deviation is given by:
\[
\Delta_{t+1} = F^{-1}\!\left( E\!\left( F(\Delta_t) \right) \right).
\]

This process can be described in three steps:
\begin{enumerate}
    \item The current deviation $\Delta_t$ is converted into the equilibrium indicator $\mathrm{FXI}$ via the mapping $F$.
    \item The corrective operator $E$ determines the target equilibrium value for the next moment.
    \item The new equilibrium value is mapped back into deviation $\Delta_{t+1}$ using $F^{-1}$.
\end{enumerate}

Biologically, this corresponds to:
\begin{itemize}
    \item the organism measuring the degree of pathology,
    \item the immune system generating a corrective reaction,
    \item the organism reaching a new physiological state as a result.
\end{itemize}

Thus, the equation above represents a universal formalization of the \textbf{immune reaction}.

\subsection{Stability and Return to Health}

If the immune operator $E$ has a \textit{contractive} property, such that:
\[
E(x) < x \quad \text{for } x > 1,
\]
and
\[
E(x) > x \quad \text{for } x < 1,
\]
then the system inevitably moves toward $\mathrm{FXI} = 1$, and therefore toward $\Delta = 0$.

This behavior corresponds biologically to:
\begin{itemize}
    \item effective immune response,
    \item successful suppression of infection,
    \item restoration of physiological stability.
\end{itemize}

A contractive operator $E$ therefore reflects a \textbf{healthy, stable immune system}.

\subsection{Loss of Stability (Disease)}

If $E$ becomes too weak, and instead of contraction produces expansion:
\[
E(x) \ge x \quad \text{for } x > 1,
\]
the system loses equilibrium.

In biological terms, this indicates:
\begin{itemize}
    \item rapid viral expansion,
    \item aggressive tumor progression,
    \item collapse of immune response,
    \item chronic inflammation,
    \item autoimmune dysregulation.
\end{itemize}

In such cases, $\Delta$ increases, $\mathrm{FXI}$ deviates further from 1, and the organism moves into a 
state of progressive pathology.

The stability properties of the operator $E$ therefore determine whether the immune system can restore 
health or whether pathology continues to intensify.

\section{Application of the Model to Various Immune Scenarios}

Because Flexionization describes the general structure of dynamic deviation and corrective response, 
its formal apparatus can be applied to a wide range of biological situations, regardless of the specific 
mechanisms of a disease. Below are four classes of scenarios that demonstrate the universality of the model.

\subsection{Viral Infections}

In viral infections, exponential pathogen growth increases the actual structural mass $Q_p$ and, 
consequently, increases the deviation $\Delta$.

The immune system forms the corrective operator $E$, which includes:
\begin{itemize}
    \item activation of innate immunity,
    \item fever response,
    \item antibody production,
    \item cytotoxic destruction of infected cells.
\end{itemize}

If $E$ acts as a contractive operator, then:
\[
\Delta_{t+1} < \Delta_t,
\]
and the system gradually returns to $\mathrm{FXI} = 1$.

If the corrective response is insufficient (as in immunodeficiency), $\Delta$ increases, and the infection 
may progress into a severe or chronic form.

\subsection{Oncology: Tumor vs.~Immune System}

A tumor process can be viewed as growth of a pathological component of $Q_p$ that no longer follows 
normal physiological constraints. In this case, $\Delta$ increases rapidly.

The immune operator $E$ includes:
\begin{itemize}
    \item recognition of tumor antigens,
    \item activation of NK cells,
    \item T-cell mediated cytotoxic response,
    \item destruction of tumor cells.
\end{itemize}

However, in oncology the operator $E$ often loses contractive behavior:
\begin{itemize}
    \item tumors suppress immune response,
    \item create immunosuppressive microenvironments,
    \item weaken cytotoxic activity.
\end{itemize}

In Flexionization terms:
\[
E(\mathrm{FXI}) \approx \mathrm{FXI} \quad \text{or} \quad E(\mathrm{FXI}) > \mathrm{FXI},
\]
leading to a loss of stability and continual increase of $\Delta$.

Thus, cancer is an example of a system in which the corrective operator $E$ becomes functionally impaired.

\subsection{Autoimmune Reactions}

In autoimmune diseases, the sign of the corrective operator $E$ becomes reversed:
\begin{itemize}
    \item the immune system attacks the organism's own tissues,
    \item the corrective mechanism increases deviation instead of reducing it.
\end{itemize}

In model terms:
\[
E(x) < x \quad \text{for } x < 1,
\]
meaning the system moves \emph{away} from structural symmetry ($\mathrm{FXI} = 1$).

This provides a formal representation of autoimmune dysregulation.

\subsection{Immunodeficiencies}

In immunodeficiencies, the operator $E$ has the correct direction but insufficient strength:
\[
0 < |\,E(x) - 1\,| \ll |\,x - 1\,|.
\]

The immune system attempts to reduce deviation but cannot provide the necessary corrective force.

As a result:
\[
\Delta_{t+1} \approx \Delta_t \quad \text{or} \quad \Delta_{t+1} > \Delta_t,
\]
leading to chronic infections, hypersensitivity, and weakened defensive capabilities.

Thus, Flexionization enables formal identification of the root causes of immune instability 
across a wide range of clinical scenarios.

\section{Advantages of the Flexionization Model for Biomedical Research}

Using Flexionization to describe immune dynamics offers several advantages that make this model 
highly promising for both theoretical research and applied biomedical applications.

\textbf{1. Universality.}  
The model does not depend on the specific nature of a disease, biochemical mechanisms, 
or particular cellular interactions.  
It captures the general dynamic principles of:  
\[
\text{deviation} \;\longrightarrow\; \text{corrective response} \;\longrightarrow\; \text{restoration of equilibrium}.
\]

\textbf{2. Formal mathematical rigor.}  
Flexionization is based on axioms, well-defined mappings, strict dynamic rules, and stability theorems.  
This allows analysis of immune behavior from the viewpoint of control theory and dynamical systems.

\textbf{3. A unified framework for diverse diseases.}  
The same structural model applies to:
\begin{itemize}
    \item viral infections,
    \item tumors,
    \item autoimmune disorders,
    \item immunodeficiency states.
\end{itemize}
Differences arise only in the properties of the corrective operator $E$.

\textbf{4. Clear interpretation of stability.}  
Contractive behavior of the operator $E$ corresponds to a healthy immune system,  
while loss of contraction corresponds to pathological conditions.  
This provides intuitive understanding of stability and disease progression.

\textbf{5. Suitability for formal modeling and simulations.}  
The dynamic equation
\[
\Delta_{t+1} = F^{-1}\!\left( E\!\left( F(\Delta_t) \right) \right)
\]
is particularly convenient for:
\begin{itemize}
    \item computational models,
    \item simulation engines,
    \item predictive algorithms.
\end{itemize}

\textbf{6. Applicability in medicine and biotechnology.}  
Flexionization can be used to:
\begin{itemize}
    \item study tumor dynamics,
    \item model therapeutic interventions,
    \item analyze vaccine responses,
    \item design diagnostic algorithms.
\end{itemize}

Thus, Flexionization provides a new mathematical language for describing the immune system, 
integrating biology, dynamical systems, and control theory into a single coherent framework.

\section{Conclusion}

This work demonstrates that the Flexionization theory can serve as a universal formal foundation 
for describing immune dynamics. The key structural elements of the model --- synthetic mass $Q_p$, 
reference mass $Q_F$, deviation $\Delta$, the equilibrium indicator $\mathrm{FXI}$, and the corrective 
operator $E$ --- naturally correspond to the biological mechanisms of the immune response.

This correspondence makes it possible to describe a wide spectrum of immune scenarios --- from viral 
infections to oncological and autoimmune processes --- within a single mathematical framework. 
The model does not depend on specific biochemical mechanisms and relies solely on the fundamental 
principles of feedback, stability, and corrective regulation.

Flexionization provides a new approach to analyzing immunity by unifying biological insight with 
the mathematical tools of dynamical systems theory. Further development of the model may include:
\begin{itemize}
    \item adapting the operator $E$ to specific diseases,
    \item constructing computational simulation engines,
    \item applying optimal control methods to model therapeutic interventions.
\end{itemize}

\section{References}

\begin{enumerate}
    \item Bogdanov M. \textit{Flexionization: Formal Theory of Dynamic Quantitative Equilibrium.}
    Zenodo Preprint, 2025. DOI: 10.5281/zenodo.17618948.

    \item Perelson A., Weisbuch G. \textit{Immunology for Physicists.}
    Reviews of Modern Physics, 1997.

    \item Nowak M., May R. \textit{Virus Dynamics: Mathematical Principles of Immunology and Virology.}
    Oxford University Press, 2000.

    \item Strogatz S. \textit{Nonlinear Dynamics and Chaos.}
    Westview Press.

    \item Khalil H. \textit{Nonlinear Systems.}
    Prentice Hall.

    \item De Boer R., Perelson A. \textit{Target Cell Limited Models of Viral Infection: 
    Mathematical Analysis and Applications.} Journal of Virology.

    \item Altan-Bonnet G., Hoppe A. \textit{Mathematical Models of T-cell Activation and Immune Regulation.}
    Nature Reviews Immunology.

    \item Bertsekas D. \textit{Dynamic Programming and Optimal Control.}
    Athena Scientific.

    \item Rockafellar R. \textit{Convex Analysis.}
    Princeton University Press.
\end{enumerate}

\end{document}
