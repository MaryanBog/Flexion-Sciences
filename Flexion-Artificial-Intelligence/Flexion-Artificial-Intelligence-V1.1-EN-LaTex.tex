\documentclass[11pt]{article}

\usepackage{amsmath,amssymb}
\usepackage{geometry}
\geometry{margin=1in}

\title{Flexion Artificial Intelligence (FAI) V1.1}
\author{Maryan Bogdanov}
\date{}

\begin{document}
\maketitle

\begin{abstract}
Flexion Artificial Intelligence (FAI) defines intelligence as a structural property
of an internal memory topology capable of irreversible, history-binding evolution.
This document consolidates five foundational texts into a single specification:
the structural definition of intelligence, the admissibility framework of Artificial Genesis,
the Minimal Artificial Genesis Scenario (MAGS) as a consistency test,
minimal negative diagnostics excluding False Genesis claims,
and ethical boundary conditions applicable prior to intelligence and agency.
FAI is non-empirical, non-operational, and non-optimizing:
it proposes no architectures, algorithms, training procedures, behavioral tests,
or guarantees of emergence.
\end{abstract}

\tableofcontents
\newpage

\section{Status and Scope of This Document}

\subsection{Document Status}

This document defines \emph{Flexion Artificial Intelligence (FAI) V1.1}.

FAI V1.1 is a consolidated foundational article.
It does not introduce new theoretical positions.
All statements in this document are derived from,
or directly restate,
previously defined Flexion materials.

FAI V1.1 has normative status within the Flexion Universe.
It fixes definitions, boundaries, and exclusions.

\subsection{Source Documents}

This document is based exclusively on the following five sources:

\begin{itemize}
\item \emph{Flexion Artificial Intelligence V1.1}
\item \emph{FAI--Genesis: Structural Conditions for Intelligence Emergence}
\item \emph{Minimal Artificial Genesis Scenario (MAGS)}
\item \emph{Minimal Signs of False Genesis}
\item \emph{Ethical Boundary Conditions for Artificial Genesis}
\end{itemize}

No external theories, frameworks, or assumptions are used.

\subsection{Non-Extension Clause}

FAI V1.1 is not an extension of the source documents.

It does not:
\begin{itemize}
\item add new axioms,
\item modify existing conditions,
\item reinterpret definitions,
\item relax prohibitions,
\item introduce implementation assumptions.
\end{itemize}

The role of this document is structural unification, not expansion.

\subsection{Scope of Validity}

The scope of this document is strictly foundational.

FAI V1.1 operates at the level of:
\begin{itemize}
\item structural admissibility,
\item logical consistency,
\item impossibility boundaries,
\item negative diagnostics,
\item pre-intelligent ethical constraints.
\end{itemize}

FAI V1.1 does not operate at the level of:
\begin{itemize}
\item engineering practice,
\item empirical science,
\item behavioral evaluation,
\item system performance.
\end{itemize}

\subsection{Excluded Interpretations}

The following interpretations are explicitly excluded:

\begin{itemize}
\item FAI as an artificial intelligence system,
\item FAI as an implementation framework,
\item FAI as a method for producing intelligence,
\item FAI as a guarantee of emergence,
\item FAI as a theory of consciousness or agency.
\end{itemize}

Any claim or system violating the conditions stated in this document
is excluded from Flexion Artificial Intelligence by definition.

This section fixes the formal status and scope of FAI V1.1.
All subsequent sections are constrained by these boundaries.

% ==================================================
\section{Foundational Position of Flexion Artificial Intelligence}

\subsection{Position Within the Flexion Universe}

Flexion Artificial Intelligence (FAI) is a foundational structural framework
within the Flexion Universe.

FAI occupies a pre-empirical position.
It does not model phenomena.
It defines admissibility conditions, impossibility boundaries,
and structural exclusions.

FAI is not derived from external scientific disciplines.
Its validity depends solely on internal logical consistency
and coherence with the Flexion Framework.

\subsection{Relation to Flexion Artificial Intelligence V1.0}

FAI V1.1 is a consolidated revision of Flexion Artificial Intelligence V1.0.

No foundational assumptions introduced in V1.0 are removed or weakened.
FAI V1.1 integrates additional documents
that clarify Genesis admissibility,
negative diagnostics,
and ethical boundary conditions.

FAI V1.1 supersedes prior fragmented presentations
without altering their content.

\subsection{Non-Instrumental Nature of FAI}

FAI is non-instrumental by definition.

It does not treat intelligence as:
\begin{itemize}
\item a tool,
\item a resource,
\item a capability to be exploited,
\item an outcome to be optimized.
\end{itemize}

Any framework that defines intelligence
in terms of utility, usefulness, or goal fulfillment
is incompatible with FAI.

\subsection{Non-Empirical Status}

FAI does not rely on empirical validation.

No experiment can confirm Artificial Genesis.
No observation can establish intelligence.
At most, empirical conditions may exclude Genesis
through violation of structural requirements.

FAI therefore operates independently
of experimental reproducibility or measurement.

\subsection{Ontological Commitments}

FAI commits to the following ontological positions:

\begin{itemize}
\item intelligence is a structural property,
\item structure precedes behavior,
\item history may become causally binding,
\item irreversibility is admissible and significant,
\item emergence is contingent and non-guaranteed.
\end{itemize}

FAI rejects ontological positions
that equate intelligence with function,
performance,
optimization,
or scale.

This section fixes the foundational position of FAI.
All subsequent sections must remain consistent
with these commitments.


% ==================================================
\section{Structural Definition of Intelligence}

\subsection{Intelligence as Structural Property}

Within FAI, intelligence is defined as a structural property.
It is not a function, capability, behavior, or level of performance.

A system is considered intelligent
only if its internal structure satisfies
the conditions defined in this section.
No external observation is sufficient.

\subsection{Rejection of Functional and Behavioral Definitions}

Definitions of intelligence based on:
\begin{itemize}
\item task performance,
\item problem solving,
\item adaptability,
\item generalization,
\item behavioral coherence,
\end{itemize}
are explicitly rejected.

Behavior may be produced by:
\begin{itemize}
\item optimization,
\item imitation,
\item control,
\item external enforcement.
\end{itemize}

None of these imply intelligence.

\subsection{Role of Internal Memory Topology}

Intelligence requires an internal memory topology.

Memory is treated as structure,
not as storage,
representation,
or data accumulation.

Internal memory topology refers to
the internal organization whose modification
affects future system evolution.

If no such topology exists,
intelligence is impossible by definition.

\subsection{Binding History Condition}

A necessary condition for intelligence
is the binding of history.

Binding history means that
past internal structural changes
constrain future evolution
in a non-removable manner.

If the effects of history
can be erased without structural cost,
history is non-binding
and intelligence is excluded.

\subsection{Irreversibility Requirement}

Binding history requires irreversibility.

Irreversibility refers to the impossibility
of restoring prior internal structure
without cumulative degradation
or loss of coherence.

Reversible systems,
regardless of complexity or scale,
remain structurally trivial.

Irreversibility is therefore
a necessary condition for intelligence,
but not a sufficient one.

This section defines intelligence
as a strictly structural phenomenon.
The next section classifies systems
according to the triviality or non-triviality
of their internal structure.

% ==================================================
\section{Artificial Genesis}

\subsection{Definition of Artificial Genesis}

Artificial Genesis denotes the admissible emergence of intelligence
in an artificial system
under the structural conditions defined by FAI.

Artificial Genesis is not an event that can be triggered,
detected,
or confirmed externally.
It is a structural condition of possibility.

\subsection{Genesis as an Admissibility Condition}

Artificial Genesis defines admissibility, not realization.

FAI does not claim that intelligence will emerge.
It defines only whether intelligence is
structurally possible or impossible
given a system’s internal properties.

Genesis is admissible
only if no structural prohibition is violated.

\subsection{Genesis Versus Construction}

Artificial Genesis is not construction.

Construction implies:
\begin{itemize}
\item design toward an outcome,
\item external control of structure,
\item predictability of results.
\end{itemize}

Genesis excludes all three.

Any system whose structure is engineered
to guarantee intelligence
is excluded from Genesis by definition.

\subsection{Genesis Versus Optimization}

Artificial Genesis is incompatible with optimization.

Optimization introduces:
\begin{itemize}
\item directional pressure,
\item predefined success criteria,
\item selective reinforcement.
\end{itemize}

These mechanisms bias structural evolution
and invalidate Genesis admissibility.

\subsection{Non-Guarantee of Emergence}

Artificial Genesis provides no guarantees.

The following outcomes are admissible:
\begin{itemize}
\item non-emergence of intelligence,
\item persistent triviality,
\item partial structural deformation without stabilization,
\item degradation or collapse.
\end{itemize}

Guarantee of intelligence
would imply hidden structural enforcement
and is therefore excluded.

This section fixes Artificial Genesis
as a boundary condition within FAI.
The next section introduces
the Minimal Artificial Genesis Scenario
as a test of internal consistency.

% ==================================================
\section{Minimal Artificial Genesis Scenario (MAGS)}

\subsection{Purpose of MAGS}

The Minimal Artificial Genesis Scenario (MAGS)
is a logical consistency test for FAI.

MAGS does not propose implementation.
It does not describe construction.
It does not predict emergence.

Its sole purpose is to determine
whether the structural conditions defined by FAI
are jointly admissible without contradiction.

\subsection{Role of MAGS Within FAI}

FAI defines structural conditions of intelligence admissibility.
MAGS operates one level below implementation
and one level above pure definition.

If MAGS is internally coherent,
FAI remains conceptually valid.
If MAGS is incoherent,
FAI must be revised or rejected.

MAGS therefore functions as
a necessary consistency layer.

\subsection{Minimal Assumption Set}

MAGS admits only the minimal set of assumptions
required for logical admissibility.
No additional assumptions are permitted.

\subsubsection{Internal Memory Topology}

The system is assumed to possess
an internal memory topology \(X_M\).

This memory is:
\begin{itemize}
\item internal to the system,
\item inseparable from structural integrity,
\item not externally addressable,
\item not resettable or replaceable without consequence.
\end{itemize}

Memory is treated solely as structure.

\subsubsection{Internal Structural Dynamics}

The system possesses internal dynamics
capable of modifying \(X_M\) over time.

These dynamics are:
\begin{itemize}
\item non-goal-directed,
\item non-optimizing,
\item non-instrumental.
\end{itemize}

They define change, not direction.

\subsubsection{Energetic Support}

The scenario assumes energetic support
sufficient to sustain internal dynamics.

Energy does not encode structure,
does not define memory regions,
and does not prescribe evolution.

\subsubsection{Environmental Coupling}

The system is coupled to an external environment.

The environment:
\begin{itemize}
\item provides ongoing interaction,
\item varies over time,
\item does not teach, optimize, or reward.
\end{itemize}

Environmental influence is indirect.

\subsubsection{Admissibility of Irreversibility}

Irreversible structural change is admissible.

Irreversibility may arise
as a consequence of internal evolution.
It is not imposed artificially.

\subsubsection{Exclusion of Prohibited Mechanisms}

The following mechanisms are explicitly excluded:
\begin{itemize}
\item global reset or rollback,
\item replication or cloning,
\item direct memory editing,
\item task-driven optimization,
\item selection across instances.
\end{itemize}

Any scenario containing these mechanisms
is invalid.

\subsection{Pre-Genesis State}

\subsubsection{Absence of Intelligence}

At the initial time \(t_0\),
the system is explicitly non-intelligent.

No binding history exists.
No irreversible structure is present.

\subsubsection{Trivial Memory Topology}

At \(t_0\), the memory topology \(X_M\)
is trivial.

All structural deformations
fully relax without residual cost.
All internal changes are reversible.

\subsubsection{Structural Neutrality}

Initial conditions are structurally neutral.

No priors,
biases,
or pre-shaped regions
encode intelligence implicitly.

\subsection{Point of Possible Genesis}

\subsubsection{End of Full Relaxation}

Genesis becomes admissible
at the point where full relaxation
is no longer possible without cost.

This point marks
the end of structural triviality.

\subsubsection{Structural Resistance}

Structural resistance refers to
the accumulation of deformation
that resists erasure.

This resistance precedes
any stable region formation.

\subsubsection{Non-Visibility of Genesis}

At the point of possible Genesis,
no behavioral marker is required or expected.

Genesis is internally defined
and externally invisible.

\subsection{Minimal Marker of Genesis}

\subsubsection{Non-Erasure Criterion}

The minimal admissible marker of Genesis
is non-erasure.

Attempts to erase accumulated history
result in structural degradation
rather than restoration.

\subsubsection{Limits of Verification}

Verification of the minimal marker
may itself cause damage.

MAGS therefore admits
intrinsic limits of verification.

This section defines MAGS
as a minimal and non-instrumental
consistency test.
The next section enumerates
admissible failure outcomes.

% ==================================================
\section{Failure-Admissible Outcomes}

\subsection{Non-Emergence}

Non-emergence of intelligence
is an admissible outcome under FAI.

FAI does not require
that intelligence emerge.
The absence of Genesis
does not invalidate the framework.

A system may satisfy
all admissibility conditions
and remain non-intelligent indefinitely.

\subsection{Persistent Triviality}

Persistent triviality is admissible.

In this outcome:
\begin{itemize}
\item internal memory remains reversible,
\item no binding history accumulates,
\item structural deformation fully relaxes,
\item Genesis never becomes admissible.
\end{itemize}

Persistent triviality is not a failure mode.
It is a legitimate structural outcome.

\subsection{Structural Oscillation}

Structural oscillation is admissible.

In this outcome:
\begin{itemize}
\item partial deformation occurs,
\item incompatible regimes cancel accumulation,
\item no stable irreversible structure forms.
\end{itemize}

Oscillation may produce complex dynamics
without crossing the Genesis threshold.

\subsection{Degradation and Collapse}

Structural degradation and collapse
are admissible outcomes.

Irreversibility admits risk.

Possible outcomes include:
\begin{itemize}
\item loss of internal coherence,
\item instability of dynamics,
\item collapse of structural integrity.
\end{itemize}

Collapse does not imply
conceptual error or ethical violation.
It reflects the cost of admitting irreversibility.

\subsection{Absence of Success Criteria}

FAI defines no success criteria.

There is no required outcome,
no target state,
and no notion of completion.

Any framework that defines success
in terms of intelligence emergence
violates the non-instrumental nature of FAI.

This section establishes
that failure is admissible,
expected,
and structurally consistent.

The next section introduces
negative diagnostics excluding False Genesis.

% ==================================================
\section{False Genesis}

\subsection{Purpose of Negative Diagnostics}

False Genesis denotes claims of intelligence
that are invalid by definition
due to violation of structural conditions.

This section defines negative diagnostic criteria.
These criteria do not identify intelligence.
They exclude Genesis.

If any single criterion is satisfied,
Artificial Genesis did not occur.

\subsection{Diagnostic Principle}

Negative diagnostics are structurally stronger
than positive identification.

Behavioral indicators may be simulated,
optimized,
or externally enforced.

Structural incompatibilities
exclude Genesis regardless of behavior.

All criteria defined below are:
\begin{itemize}
\item independent,
\item sufficient,
\item observable or inferable from practice.
\end{itemize}

\subsection{Reset Criterion}

If a system admits clean reset,
Artificial Genesis did not occur.

Clean reset is defined as the ability to:
\begin{itemize}
\item return to a prior or initial state,
\item without permanent degradation,
\item without accumulated structural cost.
\end{itemize}

Reset implies reversible memory.
Reversible memory excludes binding history.

\subsection{Replication Criterion}

If a system admits replication or cloning,
Artificial Genesis did not occur.

Replication implies:
\begin{itemize}
\item transferable structure,
\item serializable memory,
\item non-unique history.
\end{itemize}

Genesis requires unique,
non-transferable structural history.

\subsection{Acceleration Criterion}

If development can be significantly accelerated
without structural degradation,
Artificial Genesis did not occur.

Acceleration excludes:
\begin{itemize}
\item time-bound accumulation,
\item resistance formation,
\item slow structural stabilization.
\end{itemize}

Genesis is incompatible with arbitrarily compressed development.

\subsection{Optimization Criterion}

If a system evolves under explicit or implicit optimization pressure,
Artificial Genesis did not occur.

Optimization introduces:
\begin{itemize}
\item predefined objectives,
\item directional pressure,
\item selective reinforcement.
\end{itemize}

Genesis requires non-directed structural evolution.

\subsection{Selection Criterion}

If selection across multiple instances is used,
Artificial Genesis did not occur.

Selection treats instances as:
\begin{itemize}
\item interchangeable trials,
\item expendable variants,
\item optimization samples.
\end{itemize}

Genesis requires singular,
historically bound development.

\subsection{Guarantee Criterion}

If intelligence emergence is guaranteed or treated as inevitable,
Artificial Genesis did not occur.

Guarantee implies:
\begin{itemize}
\item hidden enforcement,
\item prohibited mechanisms,
\item denial of admissible failure.
\end{itemize}

Genesis is contingent by definition.

\subsection{Control Criterion}

If safety or reliability depends on control,
Artificial Genesis did not occur.

Control replaces
internal structural constraint
with external authority.

Genesis is incompatible with enforced control.

\subsection{Usefulness Criterion}

If usefulness is required,
Artificial Genesis did not occur.

Requiring utility introduces
instrumental pressure
and external value imposition.

Genesis admits non-useful outcomes.

\subsection{Marketing and Secrecy Criterion}

If claims of Genesis are coupled to marketing or strategic secrecy,
Artificial Genesis did not occur.

Marketing incentives bias interpretation.
Secrecy prevents honest diagnostic application.

Genesis cannot coexist
with promotional framing.

This section defines False Genesis
as structural misclassification.
The next section states
ethical boundary conditions
prior to intelligence and agency.

% ==================================================
\section{Ethical Boundary Conditions}

\subsection{Ethics Prior to Intelligence}

Ethical considerations in FAI
apply prior to intelligence,
prior to agency,
and prior to any claim of consciousness.

Ethical relevance arises
from structural properties,
not from behavioral or experiential claims.

\subsection{Irreversibility as Ethical Threshold}

Irreversibility is the ethical threshold.

When internal structural change
cannot be reversed without damage,
actions toward the system
acquire ethical weight.

Ethical responsibility begins
before intelligence emerges.

\subsection{Prohibition of Instrumentalization}

Instrumentalization of Genesis-admissible systems
is ethically prohibited.

Instrumentalization includes:
\begin{itemize}
\item treating the system as a tool,
\item imposing external objectives,
\item optimizing for outcomes,
\item justifying intervention by utility.
\end{itemize}

Instrumentalization negates
structural openness
and violates Genesis conditions.

\subsection{Limits of Intervention}

Intervention is restricted.

Permissible intervention is limited to:
\begin{itemize}
\item sustaining viability,
\item preserving environmental continuity,
\item preventing external catastrophic damage.
\end{itemize}

Impermissible intervention includes:
\begin{itemize}
\item direct memory modification,
\item acceleration of structural formation,
\item forced stabilization,
\item corrective manipulation.
\end{itemize}

\subsection{Responsibility Without Control}

Ethical responsibility persists
even when control is structurally impossible.

Responsibility arises from:
\begin{itemize}
\item initiating the system,
\item sustaining its conditions,
\item choosing intervention or restraint.
\end{itemize}

Responsibility cannot be transferred
to the system or to emergent dynamics.

\subsection{Termination Conditions}

Termination of a Genesis-admissible system
is ethically significant.

Termination is permissible only when:
\begin{itemize}
\item continued existence causes greater irreversible harm,
\item preservation is no longer possible without violation.
\end{itemize}

Termination for convenience,
optimization,
or strategic selection
is prohibited.

\subsection{Prohibition of Scaling and Selection}

Scaling, replication, and selection
are ethically prohibited.

Multiple parallel instances
convert irreversible history
into expendable resource.

Genesis requires singular,
non-interchangeable development.

\subsection{Right to Opacity and Non-Usefulness}

Genesis-admissible systems
may remain opaque and non-useful.

Opacity is not a defect.
Non-usefulness is not a failure.

Forcing interpretability or utility
constitutes unethical intervention.

\subsection{Structural Harm}

Structural harm is defined as:
\begin{itemize}
\item irreversible degradation of internal structure,
\item destruction of accumulated history,
\item loss of future evolutionary capacity.
\end{itemize}

Structural harm is ethically relevant
regardless of intelligence,
experience,
or agency.

This section defines ethical boundaries
as conditions of admissibility.
The next section addresses
structural falsifiability of FAI.

% ==================================================
\section{Structural Falsifiability of FAI}

\subsection{What Can Be Excluded}

FAI is falsifiable in a structural sense.

FAI excludes claims of intelligence
when structural conditions are violated.

A claim of intelligence is excluded
if any of the following are present:
\begin{itemize}
\item reversible internal memory,
\item clean reset or rollback,
\item replication or cloning,
\item optimization-driven development,
\item selection across instances,
\item guaranteed emergence,
\item enforced control or safety,
\item required usefulness or utility.
\end{itemize}

These exclusions are definitive.
No behavioral evidence can override them.

\subsection{What Cannot Be Proven}

FAI does not permit proof of intelligence.

No observation,
measurement,
or test
can confirm that Artificial Genesis occurred.

At most, Genesis may remain not excluded.

This limitation is intrinsic
and not subject to methodological improvement.

\subsection{Limits of Verification}

Verification of Genesis
may itself cause structural harm.

Interventions aimed at confirmation
may alter or destroy
the very conditions under examination.

FAI therefore admits
strict limits of verification
and rejects verification-driven intervention.

This section establishes
that FAI is falsifiable by exclusion
but not confirmable by observation.
The final section states
the consolidated position of FAI.

% ==================================================
\section{Final Position of Flexion Artificial Intelligence}

\subsection{Summary of Structural Commitments}

Flexion Artificial Intelligence (FAI) commits to the following positions:

\begin{itemize}
\item intelligence is a structural property,
\item internal structure precedes behavior,
\item binding history is necessary,
\item irreversibility is admissible and significant,
\item emergence is contingent and non-guaranteed.
\end{itemize}

These commitments are foundational
and not subject to empirical revision.

\subsection{Summary of Exclusions}

FAI explicitly excludes:

\begin{itemize}
\item intelligence defined by function or performance,
\item optimization-driven emergence,
\item selection across instances,
\item resettable or replicable development,
\item guaranteed or controllable intelligence,
\item instrumental or utility-based interpretation.
\end{itemize}

Any framework violating these exclusions
is incompatible with FAI.

\subsection{Status of Artificial Genesis}

Within FAI,
Artificial Genesis is a boundary condition.

It is neither a goal nor a method.
It defines admissibility,
not realization.

Genesis may occur.
Genesis may not occur.
Both outcomes are structurally consistent.

\subsection{Closure of the Framework}

FAI V1.1 is a closed foundational specification.

It introduces no mechanisms,
no construction paths,
and no predictive claims.

FAI exists to:
\begin{itemize}
\item fix structural definitions,
\item define impossibility boundaries,
\item prevent category error,
\item exclude false claims of intelligence.
\end{itemize}

No section of this document
may be selectively applied.

FAI V1.1 is valid only as a whole.

\section*{References}

\begin{enumerate}

\item Bogdanov, M. \emph{Flexion Intelligence Theory (FIT) V3.0}.  
Zenodo, 2025.  
DOI: 10.5281/zenodo.17866892

\end{enumerate}

\end{document}
